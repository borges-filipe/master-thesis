\chapter{Introduction\label{cha:chapter1}}

Usually the introduction chapter is separated into subsections like 'motivation', 'objective', and 'outline'.

\section{Motivation\label{sec:moti}}

The introduction should make people aware of the problem that you are trying to solve with your concept, respectively implementation.

\section{Objective\label{sec:objective}}

What kind of problem do you adress? Which issues do you try to solve? What solution do you propose? What is your goal?
'This thesis describes an approach to combining X and Y... The aim of this work is to...'

\section{Outline\label{sec:outline}}

The 'structure' or 'outline' section gives a brief introduction into the main chapters of your work. Write 2-5 lines about each chapter.
\\
\\
\noindent This example thesis is separated into 5 chapters.
\\
\\
\textbf{Chapter \ref{cha:chapter2}} is usually termed 'Related Work', 'State of the Art' or 'Fundamentals'. Here you will describe relevant technologies and standards related to your topic. What did other scientists propose regarding your topic? This chapter makes about 20-30 percent of the complete thesis.
\\
\\
\textbf{Chapter \ref{cha:chapter3}} describes the implementation part of your work. Don't explain every code detail but emphasize important aspects of your implementation. This chapter will have a volume of 15-20 percent of your thesis.
\\
\\
\textbf{Chapter \ref{cha:chapter4}} is usually termed 'Evaluation' or 'Validation'. How did you test it? In which environment? How does it scale? Measurements, tests, screenshots. This chapter will have a volume of 10-15 percent of your thesis.
\\
\\
\textbf{Chapter \ref{cha:chapter5}} summarizes the thesis, describes the problems that occurred and gives an outlook about future work. Should have about 4-6 pages.